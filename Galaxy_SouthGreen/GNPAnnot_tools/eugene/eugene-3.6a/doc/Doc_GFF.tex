% Documentation of the GFF sensor

\subsubsection{\texttt{Sensor.GFF}}

\paragraph{Description}

The GFF sensor allows to add to the graphical representation an
annotation provided in a GFF format. Note that the provided GFF
annotation could be an \EuGene prediction given in GFF format
(obtained using the $-pg$ argument). This could allow to visualise two
predictions on the same graph.

For a sequence, the plugin reads the annotation from one file whose
name is derived from the sequence name by adding the \texttt{.gff}
suffix.  The sensor is activated by either :
\begin{itemize}
\item the \texttt{-G} argument \index{CmdFlags}{[GFF activation] G}
\item the value 1 for the parameter \texttt{Sensor.GFF.use} in the
  parameter file.
\end{itemize}
Here is an example of GFF parameters definition :
\begin{Verbatim}[fontsize=\small]
Sensor.GFF.use  1      # Use GFF sensor
Sensor.GFF      1         # Sensor priority
\end{Verbatim}

\paragraph{Native input files format}

The file \texttt{.gff} describes an annotation for a sequence. The format of
a line is : \texttt{<seqname> <source> <feature> <start> <end> <score>
  <strand> <frame>}. Seqname, source and score fields are ignored.

Example:
\begin{Verbatim}[fontsize=\small]
seqName  EuGene  Utr5    1       199     0       -       .
seqName  EuGene  Utr5    340     359     0       +       .
seqName  EuGene  Init    360     393     0       +       2
seqName  EuGene  Intr    596     732     0       +       0
seqName  EuGene  Intr    830     876     0       +       1
seqName  EuGene  Intr    961     1286    0       +       1
seqName  EuGene  Intr    1396    1478    0       +       2
seqName  EuGene  Intr    1573    1648    0       +       0
seqName  EuGene  Intr    1757    1818    0       +       0
seqName  EuGene  Intr    1962    2057    0       +       2
seqName  EuGene  Intr    2145    2306    0       +       2
seqName  EuGene  Term    2491    2607    0       +       0
seqName  EuGene  Utr3    2608    2626    0       +       .
\end{Verbatim}
Note: only exons are plotted, this file is parsing by the frame field
(no `.' in the frame field).

\paragraph{Gff3 input files format}

The gff3 input mode is activated by setting the value \texttt{GFF3}
for the parameter \texttt{GFF.format} in the parameter file.  The
plugin reads the predictions from a file which name is derived from
the sequence name by adding the \texttt{.gff.gff3}

Accepted features (third column of GFF3 lines):\\
\begin{itemize}
\item  SO:0000316 or CDS
\item  SO:0000204 or five\_prime\_UTR
\item  SO:0000205 or three\_prime\_UTR
\end{itemize}

If the feature isn't one of those, the line won't be take into
account.  If you define parent link between feature , parent feature
must be define before children.  Here an extract of \texttt{seq14ac002535g4g5.tfa.gff.gff3}.
\begin{Verbatim}[fontsize=\tiny]
seq25   EuGene  five_prime_UTR  1       2787    0       +       .       ID=five_prime_UTR:seq25.0;Ontology_term=SO:0000204
seq25   EuGene  CDS     2788    2836    0       +       0       ID=CDS:seq25.1;Ontology_term=SO:0000196
seq25   EuGene  CDS     8356    8471    0       +       2       ID=CDS:seq25.2;Ontology_term=SO:0000004
seq25   EuGene  CDS     8576    8667    0       +       1       ID=CDS:seq25.3;Ontology_term=SO:0000004
seq25   EuGene  CDS     9006    9061    0       +       0       ID=CDS:seq25.4;Ontology_term=SO:0000004
seq25   EuGene  CDS     9567    9655    0       +       1       ID=CDS:seq25.5;Ontology_term=SO:0000004
seq25   EuGene  CDS     10520   10535   0       +       1       ID=CDS:seq25.6;Ontology_term=SO:0000004
seq25   EuGene  CDS     10896   11134   0       +       1       ID=CDS:seq25.7;Ontology_term=SO:0000004
seq25   EuGene  CDS     11544   12005   0       +       2       ID=CDS:seq25.8;Ontology_term=SO:0000004
seq25   EuGene  CDS     12088   12900   0       +       0       ID=CDS:seq25.9;Ontology_term=SO:0000197
seq25   EuGene  three_prime_UTR 12901   14900   0       +       .       ID=three_prime_UTR:seq25.10;Ontology_term=SO:0000205
\end{Verbatim}


\paragraph{Filtering input information}

No filter.

\paragraph{Integration of information}

This sensor does not affect prediction.

\paragraph{Post analyse}

No post analyse.

\paragraph{Graph}

Orange horizontal lines are plotted on the exon tracks.

