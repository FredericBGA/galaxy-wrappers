% Documentation of the AnnotaStruct sensor

\subsubsection{\texttt{Sensor.NcRNA}}
\label{ncrna}
\paragraph{Description}

The plugin allows to take into account information from non protein coding RNA. 
For each reading ncRNA region, it rewards the ncRNA tracks, and the ncRNA transcription start and stop signals at the region extemities.

To activate the sensor, put the number of NcRNA instances you want to create to the parameter
\texttt{Sensor.NcRNA.use} in the parameter file.

Here is an example of NcRNA parameters definition:
\begin{Verbatim}[fontsize=\small]
NcRNA.FileExtension[0]  ncrna
NcRNA.NpcRna*[0]        1 
NcRNA.TStartNpc*[0]     1
NcRNA.TStopNpc*[0]      1	
NcRNA.format[0]         GFF3 # Mandatory

Sensor.NcRNA.use        1
Sensor.NcRNA            1
\end{Verbatim}


\paragraph{Gff3 input files format}

The gff3 input mode is activated by setting the value \texttt{GFF3}
for the parameter \texttt{NcRNA.format} in the parameter file.
Note that for the moment, it is mandatory because gff3 is the only reading input format.
The plugin reads its information from the file whose name is the concatenation
of the sequence file name, the \texttt{NcRNA.FileExtension} parameter 
and the \texttt{.gff3} extension.
The accepted features (third column) are:
\begin{itemize}
 \item SO:0000655 or ncRNA
 \item all the terms derived from SO:0000655. For instance, SO:0000252 or rRNA, SO:0000253 or tRNA. 
For details, see  \texttt{http://www.sequenceontology.org/wiki/index.php/Category:} \texttt{SO:0000655\_!\_ncRNA}.
\end{itemize}

If the feature used isn't one of those, the line will be rejected.

Here an extract of \texttt{seq14ac002535g4g5.tfa.ncrna.gff3}:
\begin{Verbatim}[fontsize=\tiny]
seq14	tRNAscan-SE	tRNA	4809	6126	.	+	.	ID=trna:seq14.5;
seq14	tRNAscan-SE	tRNA	7000	7100	.	+	.	ID=trna:seq14.5;
\end{Verbatim}

\paragraph{Filtering input information}

No filtering.

\paragraph{Integration of information}

At each position of a ncRNA region, the ncRNA content edge is reweighted according to the 
\texttt{NcRNA.NpcRna*} parameter value. ncRNA transcription start and stop signals 
are generated at each respective extremity according to the \texttt{NcRNA.TStartNpc*} and \texttt{NcRNA.TStopNpc*} parameter values.


\paragraph{Post analyse}

No post analyse.

\paragraph{Graph}

No plotting.










