\documentclass[a4paper,11pt]{article}

\usepackage{times}
\usepackage[french]{babel}
\usepackage{a4wide}
\usepackage{longtable}

\title{Projet EuG\`en\kern-0.09em\raise1ex\hbox{\small\it i}\kern-0.11em\hbox{e}\\Fiches projet}

\begin{document}
\maketitle

\section{Les fiches du projet}

\begin{itemize}
\item Le Quoi: le r\'esultat \`a atteindre.
\item Le Quand: une date de fin (quand un mois est indique, c'est le
  premier du mois).
\item La difficult\'e: combien de jours de travail non STOP pour
  r\'ealiser la t\^ache.
\item La priorit{\'e}: l'importance actuelle du r{\'e}sultat
  d{\'e}crit (entre 0 et 1, si c'est 1 c'est crucial).
\item Le Qui: le sous-groupe charge d'atteindre le resultat.  Le
  premier ``Qui'' est le ``task leader''. Mise place de sous groupes.
\item Statut: l'\'etat de la fiche (achev\'ee, en cours, en attente,
  annul\'ee\ldots)
\end{itemize}


\section{Fiches actives}

\begin{longtable}{|l|p{7cm}|}\hline
  Quoi & Sp\'ecifier l'ajout de la pr\'ediction d'\'epissage alternatif dans
  l'algorithme de PD. \\\hline
  Quand  &  Oct 2003\\\hline
  Difficult\'e & 2 mois \\\hline
  Priorit\'e &  0.8\\\hline
  Qui & Responsable: SF . Soutien de TS\\\hline
  Statut & \\\hline
\end{longtable}

\begin{longtable}{|l|p{7cm}|}\hline
  Quoi & Cr\'eation d'un sensor bas\'e sur des MTD acides amin\'es.\\\hline
  Quand  &  Sep 2003\\\hline
  Difficult\'e &  1 mois\\\hline
  Priorit\'e &  0.9 \\\hline
  Qui & Responsable: Xavier. Soutien de TS/SF\\\hline
  Statut & \`A faire\\\hline
\end{longtable}

\begin{longtable}{|l|p{7cm}|}\hline
  Quoi & sim2eugene: finir\\\hline
  Quand  &  Septembre 2003\\\hline
  Difficult\'e &  une semaine \\\hline
  Priorit\'e &  0.9\\\hline
  Qui & Responsable: PB.\\\hline
  Statut & \\\hline
\end{longtable}

\begin{longtable}{|l|p{7cm}|}\hline
  Quoi & Am\'eliorer/\'Evaluer EuG\`ene Riz sur la base des mRNA.\\\hline
  Quand  & Permanent\\\hline
  Difficult\'e & $n$ semaines \\\hline
  Priorit\'e & 1 (absolument indispensable, li\'e au projet) \\\hline
  Qui & Responsable: PB. Soutien de SF, TS, AM\\\hline
  Statut & En attente des s\'equences tests\\\hline
\end{longtable}

\begin{longtable}{|l|p{7cm}|}\hline
  Quoi & Documentation utilisateur au format LaTeX\\\hline
  Quand  &  tout le temps\\\hline
  Difficult\'e &  permanente\\\hline
  Priorit\'e &  0.9 \\\hline
  Qui & Responsable: TS. Soutien de Tous\\\hline
  Statut & En cours\\\hline
\end{longtable}

\begin{longtable}{|l|p{7cm}|}\hline
  Quoi & Revoir l'architecture (fichiers/classes/echanges).\\\hline
  Quand  &  Dec. 2002\\\hline
  Difficult\'e &  1 mois\\\hline
  Priorit\'e &  0.9 \\\hline
  Qui & Responsable: MJC. Soutien de TS\\\hline
  Statut & A entamer. \\\hline
\end{longtable}

\begin{longtable}{|l|p{7cm}|}\hline
  Quoi & Pr{\'e}senter l'algo de PD et BackP.cc, BackP.h \\\hline
  Quand  &  Nov. 2002\\\hline
  Difficult\'e & deux jours  \\\hline
  Priorit\'e &  0.9\\\hline
  Qui & Responsable: TS. Soutien de TS\\\hline
  Statut & En cours\\\hline
\end{longtable}


\section{Fiches \`a entamer}

\begin{longtable}{|l|p{7cm}|}\hline
  Quoi & Analyser le comportement des 2 matrices GC\% distinctes\\\hline
  Quand  &  Septembre 2003 \\\hline
  Difficult\'e & deux semaines \\\hline
  Priorit\'e &  0.9\\\hline
  Qui & Responsable: AM. Soutien de TS\\\hline
  Statut & \\\hline
\end{longtable}

\begin{longtable}{|l|p{7cm}|}\hline
  Quoi & EuGene.par est trop long ! On pourrait faire un include ou le couper en 2 (optimsation/Eugene)\\\hline
  Quand  &  Dec 2003\\\hline
  Difficult\'e & 1 sem \\\hline
  Priorit\'e &  0.6\\\hline
  Qui & Responsable: PB. Soutien de TS\\\hline
  Statut & \\\hline
\end{longtable}

\begin{longtable}{|l|p{7cm}|}\hline
  Quoi Modifier d\'es\'equilibre des tailles de jeux d'apprentissage IG/Intron& \\\hline
  Quand  &  Septembre 2003\\\hline
  Difficult\'e & 2 jours \\\hline
  Priorit\'e &  0.9 \\\hline
  Qui & Responsable: TS\\\hline
  Statut & \\\hline
\end{longtable}

\begin{longtable}{|l|p{7cm}|}\hline
  Quoi & Estimer deux fichiers .par pour les 2 GC\%\ du Riz\\\hline
  Quand  &  Septembre 2003\\\hline
  Difficult\'e &  une semaine\\\hline
  Priorit\'e &  \\\hline
  Qui & Responsable: PB.\\\hline
  Statut & \\\hline
\end{longtable}

\begin{longtable}{|l|p{7cm}|}\hline
  Quoi & Essayer ATG Pred seul\\\hline
  Quand  &  Septembre 2003\\\hline
  Difficult\'e &  une semaine \\\hline
  Priorit\'e &  0.9\\\hline
  Qui & Responsable: PB.\\\hline
  Statut & \\\hline
\end{longtable}

\begin{longtable}{|l|p{7cm}|}\hline
  Quoi & EuGene$\Omega$: ajouter RepeatMasker\\\hline
  Quand  & Sep 2003 \\\hline
  Difficult\'e & une semaine \\\hline
  Priorit\'e & 0.5 \\\hline
  Qui & Responsable: SF. Soutien de PB\\\hline
  Statut & ouvert\\\hline
\end{longtable}

\begin{longtable}{|l|p{7cm}|}\hline
  Quoi & EuGene$\Omega$: int\'egrer le nouveau parser Blast de Laurent\\\hline
  Quand  & Oct 2003 \\\hline
  Difficult\'e & une semaine \\\hline
  Priorit\'e & 0.5 \\\hline
  Qui & Responsable: SF. Soutien de PB\\\hline
  Statut & ouvert\\\hline
\end{longtable}

\begin{longtable}{|l|p{7cm}|}\hline
  Quoi & EuG\`ene$\Omega$: rajouter l'homme aux mod\`eles\\\hline
  Quand  & Fin 2003 \\\hline
  Difficult\'e &  deux semaines\\\hline
  Priorit\'e & 0.5 \\\hline
  Qui & Responsable: SF. Soutien de PB\\\hline
  Statut & ouvert\\\hline
\end{longtable}

\begin{longtable}{|l|p{7cm}|}\hline
  Quoi & Mettre \`a jour le site WEB EuG\`ene (http://www.inra.fr/bia/T/EuGene/). Le site est sur ossau dans /home/w3/T/EuGene\\\hline
  Quand  &  Des que possible \\\hline
  Difficult\'e &  2 jours \\\hline
  Priorit\'e &  0.8 \\\hline
  Qui & Responsable: PB. Soutien de SF/TS\\\hline
  Statut & a entamer\\\hline
\end{longtable}

\begin{longtable}{|l|p{7cm}|}\hline
  Quoi & Exploiter le Sensor Tester pour faire des ROC automatiques (gnuplot)\\\hline
  Quand  &  Sep. 2003\\\hline
  Difficult\'e &  1 semaine \\\hline
  Priorit\'e &  0.8\\\hline
  Qui & Responsable: AM. Soutien de TS\\\hline
  Statut & Ouvert\\\hline
\end{longtable}

\begin{longtable}{|l|p{7cm}|}\hline
  Quoi & Faire une remise a l'\'echelle des Sensors sur la base d'une analyse de la proba conditionnelle (type Res. Bay\'esien)\\\hline
  Quand  &  avant fin 2003\\\hline
  Difficult\'e &  un mois\\\hline
  Priorit\'e &  0.5 \\\hline
  Qui & Responsable: TS. Soutien de PB\\\hline
  Statut & Ouvert\\\hline
\end{longtable}

\begin{longtable}{|l|p{7cm}|}\hline
  Quoi & Cr\'eer un Sensor ``s\'equences homologues'' bas\'e sur la
  plus grande variabilit\'e de la 3\`eme position des codons.\\\hline
  Quand  & Nov. 2002 \\\hline
  Difficult\'e &  2 semaines\\\hline
  Priorit\'e &  0.9 \\\hline
  Qui & Responsable: TS. Soutien de AM, SF\\\hline
  Statut & En r\'eflexion.\\\hline
\end{longtable}

\begin{longtable}{|l|p{7cm}|}\hline
  Quoi & Inclure l'optimisation des parametres dans EuG\`ene\\\hline
  Quand  &  Sep. 2003\\\hline
  Difficult\'e & 1 mois \\\hline
  Priorit\'e & 0.7\\\hline
  Qui & Responsable: MJC. Soutien de SF/TS\\\hline
  Statut & a faire.\\\hline
\end{longtable}

\section{Fiches futures ou optionnelles}

\begin{longtable}{|l|p{7cm}|}\hline
  Quoi & Complexifier le type ``REAL'' en un co\^ut alg\'ebrique
  (eg. lexicaux a deux niveaux) pour la PD\\\hline
  Quand  & F\'ev. 2003 \\\hline
  Difficult\'e &  2 semaines\\\hline
  Priorit\'e &  0.9\\\hline
  Qui & Responsable: TS. Soutien de SF\\\hline
  Statut & \`A faire\\\hline
\end{longtable}

\begin{longtable}{|l|p{7cm}|}\hline
  Quoi & Sensor Mod\`ele de Markov glissant ADN \\\hline
  Quand  &  Mars 2003 \\\hline
  Difficult\'e &  2 mois \\\hline
  Priorit\'e &  0.8 \\\hline
  Qui & Responsable: TS. Soutien de SF\\\hline
  Statut & \\\hline
\end{longtable}

\begin{longtable}{|l|p{7cm}|}\hline
  Quoi & Sensor Mod\`ele de Markov robuste (entropie max de l'usage du codon)\\\hline
  Quand  &  Mars 2003\\\hline
  Difficult\'e &  2 mois\\\hline
  Priorit\'e &  0.8\\\hline
  Qui & Responsable: TS. Soutien de SF\\\hline
  Statut & \\\hline
\end{longtable}

\begin{longtable}{|l|p{7cm}|}\hline
  Quoi & Concevoir et \'ecrire un sensor ``fusionneur de sensor''
  (Bay\'esien naif ou non par exemple). D\'ecrire EuG\`ene comme un r\'eseau
  bay\'esien dynamique\\\hline
  Quand  &  Mai 2003\\\hline
  Difficult\'e & 1 mois \\\hline
  Priorit\'e &  0.9\\\hline
  Qui & Responsable: SF. Soutien de TS \\\hline
  Statut & \`A faire \\\hline
\end{longtable}
 
\begin{longtable}{|l|p{7cm}|}\hline
  Quoi & Remise \`a jour des pr\'ediction en cas de modification
  incr\'ementale des sorties sensors\\\hline
  Quand  &  Non born\'e\\\hline
  Difficult\'e &  un an \\\hline
  Priorit\'e &  0.6\\\hline
  Qui & Responsable: TS. Soutien de Tous\\\hline
  Statut & Tout est \`a faire\\\hline
\end{longtable}

\begin{longtable}{|l|p{7cm}|}\hline
  Quoi & Sp\'ecification d'une interface entre Eugene C++ et une interface
Java (par exemple)\\\hline
  Quand  &  Avril 2003\\\hline
  Difficult\'e & 1 mois \\\hline
  Priorit\'e &  0.7\\\hline
  Qui & Responsable: PB. Soutien de MJC\\\hline
  Statut & non ouvert\\\hline
\end{longtable}

\begin{longtable}{|l|p{7cm}|}\hline
  Quoi & Conception/impl\'ementation de l'interface Java\\\hline
  Quand  &  Sep. 2003\\\hline
  Difficult\'e & 3 mois \\\hline
  Priorit\'e &  0.7\\\hline
  Qui & Responsable: PB. Soutien de MJC, AM, TS.\\\hline
  Statut & non ouvert \\\hline
\end{longtable}

\begin{longtable}{|l|p{7cm}|}\hline
  Quoi & Specification, conception d'un sensor ``pattern matcher'' AA/ADN \\\hline
  Quand  &  Mars 2003 ou M. Cottevieille\\\hline
  Difficult\'e &  2 mois\\\hline
  Priorit\'e &  0.8 \\\hline
  Qui & Responsable: MC. Soutien de CG, TS.\\\hline
  Statut & Non ouvert\\\hline
\end{longtable}

\begin{longtable}{|l|p{7cm}|}\hline
  Quoi & Possibilit\'e de sensors en langages interpretes (Perl, Python... ?)
comme dans Gimp \\\hline
  Quand  &   Mars 2003 \\\hline
  Difficult\'e &  3 jours\\\hline
  Priorit\'e & 0.6 \\\hline
  Qui & Responsable: TS. Soutien de PB\\\hline
  Statut & non ouvert\\\hline
\end{longtable}

\begin{longtable}{|l|p{7cm}|}\hline
  Quoi & Sensor Aligneur episs{\'e} moyen fin (a l'Utopia) version cDNA\\\hline
  Quand  &  Sep 2003\\\hline
  Difficult\'e &  2 mois\\\hline
  Priorit\'e &  0.8\\\hline
  Qui & Responsable: TS. Soutien de SF\\\hline
  Statut & non ouvert\\\hline
\end{longtable}

\begin{longtable}{|l|p{7cm}|}\hline
  Quoi & Sensor Aligneur episs{\'e} moyen fin (a l'Utopia) version cDNA non
cognat/proteines \\\hline
  Quand  &  Jan. 2004\\\hline
  Difficult\'e &  3 mois\\\hline
  Priorit\'e &  0.8\\\hline
  Qui & Responsable: TS. Soutien de SF\\\hline
  Statut & non ouvert\\\hline
\end{longtable}

\begin{longtable}{|l|p{7cm}|}\hline
  Quoi & Prendre en compte des p\'enalites de longueur ?\\\hline
  Quand  &  Fin 2003\\\hline
  Difficult\'e &  un mois \\\hline
  Priorit\'e & 0.5 \\\hline
  Qui & Responsable: TS\\\hline
  Statut & non ouvert\\\hline
\end{longtable}

\begin{longtable}{|l|p{7cm}|}\hline
  Quoi & Structure du graphe modifiable par l'utilisateur: specifier
langage/algo \\\hline
  Quand   & Mai 2004 \\\hline
  Difficult\'e & 1 mois \\\hline
  Priorit\'e &  0.5\\\hline
  Qui & Responsable: TS. Soutien de SF\\\hline
  Statut & non ouvert\\\hline
\end{longtable}

\begin{longtable}{|l|p{7cm}|}\hline
  Quoi & Faire un sensor WAM STop\\\hline
  Quand  &  Dec 2003\\\hline
  Difficult\'e & une semaine\\\hline
  Priorit\'e &  0.6\\\hline
  Qui & Responsable: SF  Soutien de TS\\\hline
  Statut & non ouvert\\\hline
\end{longtable}

\section{Fiches achev\'ees}
 
\begin{longtable}{|l|p{7cm}|}\hline
  Quoi & Introduire une classe ``pr\'ediction'' pour remplacer Choice, Fixer Output.cc pour qu'il traite les frameshits.\\\hline
  Quand  &  Nov. 2002\\\hline
  Difficult\'e &  1 semaines\\\hline
  Priorit\'e & 0.99 \\\hline
  Qui & Responsable: PS. Soutien de TS\\\hline
  Statut & reste a peufiner Output.cc pour les frameshifts. Sortie graphique ?\\\hline
\end{longtable}

\begin{longtable}{|l|p{7cm}|}\hline
  Quoi & \'Ecrire des sensors de signaux PWM/WAM g\'en\'eraux\\\hline
  Quand  &  Avr. 2003\\\hline
  Difficult\'e &  2 semaines\\\hline
  Priorit\'e &  0.8 \\\hline
  Qui & Responsable: PB. Soutien de TS, SF\\\hline
  Statut & \`A faire\\\hline
\end{longtable}

\begin{longtable}{|l|p{7cm}|}\hline
  Quoi & Ajouter les frameshifts au mod\`ele. Avec une \'eventuelle
  p\'enalite locale (via les Sensors). \\hline
  Quand  &  D\'ecembre 2002 \\\hline
  Difficult\'e &  2 semaines\\\hline
  Priorit\'e &  0.8\\\hline
  Qui & Responsable: TS. Soutien de SF,PB\\\hline
  Statut & done.\\\hline 
\end{longtable}

\begin{longtable}{|l|p{7cm}|}\hline
  Quoi & Cr\'eation d'un sensor bas\'e sur des mod\`eles de Markov acides amin\'es.\\\hline
  Quand  &  Nov. 2002\\\hline
  Difficult\'e &  2 semaines\\\hline
  Priorit\'e &  0.9 \\\hline
  Qui & Responsable: SF. Soutien de TS\\\hline
  Statut & En cours\\\hline
\end{longtable}
 
\begin{longtable}{|l|p{7cm}|}\hline
  Quoi & Voir comment les Sensors (v1) peuvent passer en plugins
  chargeables dynamiquement avec dlopen\ldots.\\\hline
  Quand  &  Janvier 2003\\\hline
  Difficult\'e &  2 \`a 3 semaines\\\hline
  Priorit\'e &  0.8\\\hline
  Qui & Responsable: TS. Soutien de PB \\\hline
  Statut & En cours.\\\hline
\end{longtable}

\begin{longtable}{|l|p{7cm}|}\hline
  Quoi & D\'efinir des r\`egles de d\'eveloppement minimales (XP/RUP,Style C++)\\\hline
  Quand  & Nov. 2002 \\\hline
  Difficult\'e & 1 mois \\\hline
  Priorit\'e &  0.9 \\\hline
  Qui & Responsable: MJC. Soutien de TS \\\hline
  Statut & En cours. Le style est fix\'e. L'utilisation de XP est en
  cours d'adaptation.\\\hline
\end{longtable}

\begin{longtable}{|l|p{7cm}|}\hline
  Quoi & \'Economiser la m\'emoire: \'economie des tableaux de taille egale a celle de la s\'equence dans les sensors et Choice \\\hline
  Quand  &  Nov. 2002\\\hline
  Difficult\'e &  2 semaines\\\hline
  Priorit\'e & 0.9 \\\hline
  Qui & Responsable: PS. Soutien de TS\\\hline
  Statut & a faire\\\hline
\end{longtable}

\begin{longtable}{|l|p{7cm}|}\hline
  Quoi & Introduire une classe ``Param\`etres'' charg\'ee de lire le .par
  et de distribuer le contenu \`a Eug\`ene et aux plugins.  \\\hline
  Quand  &  Oct. 2002 \\\hline
  Difficult\'e &  1 semaine \\\hline
  Priorit\'e &  0.95 \\\hline
  Qui & Responsable: PB. Soutien de TS\\\hline
  Statut & Effectu\'e\\\hline
\end{longtable}

\begin{longtable}{|l|p{7cm}|}\hline
  Quoi & \'Eviter les probl\`emes de perte de pr\'ecision en flottant dans
  l'algo de PD.\\\hline
  Quand  &  D\'ec. 2002\\\hline
  Difficult\'e & 1 semaine \\\hline
  Priorit\'e &  0.9 \\\hline
  Qui & Responsable: TS. Soutien de personne (snif !)\\\hline
  Statut & Effectu\'e\\\hline
\end{longtable}

\begin{longtable}{|l|p{7cm}|}\hline
  Quoi & Cr\'eer un Sensor ``s\'equences homologues'' bas\'e sur les scores
  de la matrice de similarit\'e \\\hline
  Quand  & Nov. 2002 \\\hline
  Difficult\'e &  2 semaines\\\hline
  Priorit\'e &  0.9 \\\hline
  Qui & Responsable: SF. Soutien de AM\\\hline
  Statut & En cours\\\hline
\end{longtable}

\begin{longtable}{|l|p{7cm}|}\hline
  Quoi & Sensor Mod\`ele de Markov a GC\% multiples\\\hline
  Quand  &  Dec. 2002 \\\hline
  Difficult\'e &  2 semaines \\\hline
  Priorit\'e &  0.9 \\\hline
  Qui & Responsable: TS. Soutien de PB\\\hline
  Statut & A faire.\\\hline
\end{longtable}
\end{document}

\begin{longtable}{|l|p{7cm}|}\hline
  Quoi & \\\hline
  Quand  &  \\\hline
  Difficult\'e & \\\hline
  Priorit\'e &  \\\hline
  Qui & Responsable:  Soutien de TS\\\hline
  Statut & \\\hline
\end{longtable}
