% Documentation of the FrameShift sensor

\subsubsection{\texttt{Sensor.FrameShift}}

\paragraph{Description}

This plugin predicts possible frameshifts (either insertions or
deletions) at each position of the sequence with a uniform cost. The
parameters \texttt{FrameShift.Ins*} \texttt{FrameShift.Del*} give the
corresponding penalties.

The sensor is activated by either:
\begin{itemize}
\item the \texttt{-f} argument \index{CmdFlags}{[FrameShift activation] f}.
\item the value 1 for the parameter \texttt{Sensor.FrameShift.use} in
  the parameter file.
\end{itemize}

Here is an example of FrameShift parameters definition.
\begin{Verbatim}[fontsize=\small]
FrameShift.Ins*        1e999.0
FrameShift.Del*        1e999.0
Sensor.FrameShift.use  1     # Use FrameShift sensor
Sensor.FrameShift      1        # Sensor priority
\end{Verbatim}

\paragraph{Input files format}

No input files  needed.

\paragraph{Integration of information}

All predictions that use a frameshift (going from one coding phase to
another coding phase) are given an extra \texttt{FrameShift.Ins*} or
\texttt{FrameShift.Del*} penalty according to the phase change.

\paragraph{Post analyse}

No post analyse.

\paragraph{Graph}

Every predicted frameshift is plotted as a vertical red line that
connect the exonic prediction blocks in the 2 corresponding phase.


