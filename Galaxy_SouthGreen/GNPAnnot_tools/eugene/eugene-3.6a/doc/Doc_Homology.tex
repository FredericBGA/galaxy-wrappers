% Documentation of the Homology sensor

\subsubsection{\texttt{Sensor.Homology}}

\paragraph{Description}

This sensor is intended to take into account information from one or
more homologous DNA sequences, usually genomic sequences from other
species (inter-genomic homology), other sequences from the same genome
(intra-genomic homology), or transcript sequences from normalized cDNA
sets. The underlying general idea is that during evolution, functional
genomic regions (\emph{e.g.} exons) tend to be more conserved than
non-functional ones (\emph{e.g.} introns). The sensor increases the
coding score of a genomic position that is included in a conserved
region.

Homology detection has to be performed beforehand by the
\texttt{TBlastX} software. Resulting alignment files require a
specific format (see below).

The sensor is activated by either:
\begin{itemize}
\item the \texttt{-t} argument \index{CmdFlags}{[Homology activation] t}
\item a value higher than 0 for the parameter \texttt{Sensor.Homology.use} in the
  parameter file.
\end{itemize}
Here is an example of Homology parameters definition:
\begin{Verbatim}[fontsize=\small]
Homology.TblastxP*[0]     0
Homology.TblastxB*[0]     0.0595
Homology.protmatname[0]   BLOSUM80
Homology.MaxHitLen[0]     15000
Homology.FileExtension[0] .tblastx
Sensor.Homology.use       1         # Use Homology sensor
Sensor.Homology           1         # Sensor priority

\end{Verbatim}

\texttt{Homology.protmatname} is the name of the file containing the
amino acid substitution score matrix used to measure the base homology
score at the proteic level, before scaling (standard text format).
\texttt{Homology.TblastxP} and \texttt{Homology.TblastxB} parameters
are used to scale the information given by homology regions. For more
details, please refer to the publication ``\textsc{EuG\`ene'Hom}: a
generic similarity-based gene finder using multiple homologous
sequences'' (Foissac \emph{et. al}, \emph{Nucleic Acids Res.}, 2003,
31(13):3742-5).

\paragraph{Native input files format}

The plugin reads information from the file whose name is the concatenation 
of the sequence file name and the \texttt{Homology.FileExtension} parameter. 
The file describes a collection of similarities sorted by subject sequence 
and by position on the query sequence.

These files are obtained by blasting with \texttt{TBlastx} the genomic
sequence (the query) against a set of other DNA sequences, and by
parsing the results with the \texttt{ParseBlastXML.pl} script.

The TBlastX is launched with the command:

\begin{Verbatim}[fontsize=\small]
blastall -p tblastx -i QUERY_GENOMIC_SEQUENCE_FASTA -d
DATABASE_MULTIFASTA_FILE -F T -M SUBSTITUTION_MATRIX_FILE -e 1e-6 -b
50000 -m 7 > TEMPORY_BLAST_RESULT_FILE
\end{Verbatim}

Note: in order to reduce the number of ``phantom frame'' hits, the
amino acid substitution matrix (e.g. BLOSUM62) should be modified by
setting every score involving a STOP codon (lines and column noted
with a star *) to a huge penalty (e.g. -500).

and the final \texttt{.tblastx}$<i>$ files are obtained with:
\begin{Verbatim}[fontsize=\small]
ParseBlastXML.pl TEMPORY_BLAST_RESULT_FILE > QUERY_GENOMIC_SEQUENCE_FASTA.tblastx
\end{Verbatim}

For more explanation, see the README file in the directory\\
\texttt{eugene/src/SensorPlugins/Homology/GetData}.\\

One similarity S is described per line.

The format of a line is similar to the ``.blast'' file format (BlastX
sensor) with an additional column displaying the translated sequence
(amino acid alphabet) of the subject matching region:

\texttt{<$b^S$> <$e^S$> <$s^S$> <$v^S$> <$p^S$> <subject seq. name> <$bp^S$> <$ep^S$> <AA\_SEQ>}

where:
\begin{itemize}
\item $b^S$ and $e^S$ are the begin and the end of the similarity S on the query sequence,
\item $s^S$ is the score of the similarity S,
\item $v^S$ is the e-value given by TBlastX and ignored by \EuGene,
\item $p^S$ is the phase: +1, +2, +3, -1, -2, -3,
\item $bp^S$ and $ep^S$are the begin and the end of the similarity S on the subject sequence,
\item AA\_SEQ is the amino acid sequence translated from the subject nucleic region.
\end{itemize}

Here is an example of the format:
\begin{Verbatim}[fontsize=\small]
831 878 51 7e-26 -3 ATHA10A_809_856 809 856 TLQLHGRRYVETTVFV
828 878 48 1e-20 +3 ATHA10A_806_856 806 856 RDKHRCFHVSSAMKLEG
1572 1652 109 7e-114 -3 ATHA10A_1349_1429 1349 1429 IPWSNLLELKSTPMILEAPAILAPSAA
1738 1821 108 1e-153 +1 ATHA10A_1493_1576 1493 1576 FDMLLAAKEFGVTECVNPKDHDKPIQQV
830 877 86 1e-153 +2 ATHA10A_808_855 808 855 GQTPLFPRIFGHEAGG
590 625 44 1e-20 +2 ATHA10A_562_597 562 597 LQLLWHGKPESH
\end{Verbatim}

\paragraph{Gff3 input file format}

The gff3 input mode is activated by setting the value \texttt{GFF3}
for the parameter \texttt{Homology.format} in the parameter file. The
plugin reads its information from the file whose name is the concatenation
of the sequence file name, the \texttt{Homology.FileExtension} parameter 
and the \texttt{.gff3} extension. Each line reads as follows:

\texttt{<sequence name> <program> <sofa feature> <$b^S$> <$e^S$> <$v^S$> <strand> <.> 
ID=...;Target=<est name> <$bp^S$> <$ep^S$> <target strand>;frame\_hit=<$p^S$>;score\_hit=<$s^S$>}

Gff3 attributes specifications:\\
Required attributes:
	\begin{itemize}
	\item ID
	\item Target
	\item target\_sequence : amino acid sequence.
	\end{itemize}

Optional attributes:
	\begin{itemize}
	\item frame\_hit represents the blast frame. This information
          can be omitted, but if it's specified, eugene will check it
          is consistent with the similarity positions and reject the
          information otherwise.
	\item score\_hit : normalized score.
	\end{itemize}


Here an extract of \texttt{seq14ac002535g4g5.tfa.tblastx.gff3}:
\begin{Verbatim}[fontsize=\tiny]
seq14	tblastx	match	129	236	2e-25	+	.	ID=tblastx:seq14.1;Dbxref=tblastx:ATKIN2_100_207;Target=ATKIN2_100_207 100 207 +;frame_hit=+3;score_hit=118;target_sequence=KNVRDQQECLPSRSGRWQS*GTLSLLEQSTDRLFKL;
seq14	tblastx	match	86	130	3e-63	-	.	ID=tblastx:seq14.2;Dbxref=tblastx:ATKIN2_102_58;Target=ATKIN2_102_58 102 58 +;frame_hit=-2;score_hit=74;target_sequence=FQIFFSCKIVFDVCF;
\end{Verbatim}

\paragraph{Filtering input information}

Note that similarities which have a length higher than 
\texttt{Homology.MaxHitLen} parameter value are rejected. A message 
``Similarity of extreme length rejected. Check tblastx file <NAME>'' 
is printed to alert the user.

\paragraph{Integration of information}

The TBlastX search returns a set of High Scoring Pairs (HSP), each in
a given frame. All pairs of HSP which overlap and are in the same
frame are clustered together in so-called ``HSP contigs''. To
associate an homology score to a given nucleotide $n_i$ in the context
of a coding region, we consider (if it exists), the single HSP contig
$HC$ that overlaps the nucleotide in the sequence and which is in the
same frame. Let $n$ be the maximum number of HSPs in the cluster that
overlap a single position. Let $c(n_i)$ be the codon that contains
$n_i$ in the sequence, for each HSP $h$ in the cluster that overlaps
the codon $c(n_i)$, we define $S(c(n_i),h)$ to be the matrix
substitution score for the amino acid coded by $c(n_i)$ in the HSP
alignment. This score is considered as equal to zero for non
overlapping HSP. The homology score for the nucleotide in the context
of the coding region considered is defined as:
\[HS(n_i) = \frac{1}{n}\sum_{h\in HC} S(c(n_i),h)\]
The resulting score provided by the sensor after scaling is equal to
\[\mbox{\texttt{Homology.TblastxB}}\ . HS(n_i) + \mbox{\texttt{Homology.TblastxP}}\]

\paragraph{Post analyse}

None.

\paragraph{Graph}

HSP clusters are represented as grey blocks whose thickness is
proportional to the number of hits at a given position and whose
darkness is proportional to the homology score at this position.
